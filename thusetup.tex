% !TeX root = ./main.tex

% 论文基本信息配置

\thusetup{
  %******************************
  % 注意:
  %   1. 配置里面不要出现空行
  %   2. 不需要的配置信息可以删除
  %******************************
  %
  % 标题
  %   可使用“\\”命令手动控制换行
  %
  title  = {题目},
  title* = {Your title},
  include-spine = false,
}

% 载入所需的宏包

% 定理类环境宏包
\usepackage{amsthm}
% 也可以使用 ntheorem
% \usepackage[amsmath,thmmarks,hyperref]{ntheorem}

\thusetup{
  %
  % 数学字体
  % math-style = GB,  % GB | ISO | TeX
  math-font  = xits,  % sitx | xits | libertinus
}

% 可以使用 nomencl 生成符号和缩略语说明
% \usepackage{nomencl}
% \makenomenclature

% 表格加脚注
\usepackage{threeparttable}

% 表格中支持跨行
\usepackage{multirow}

% 固定宽度的表格。
\usepackage{tabularx}

% 跨页表格
\usepackage{longtable}

% 算法
% \usepackage{algorithm}
% \usepackage{algorithmic}
\usepackage[ruled, vlined, linesnumbered]{algorithm2e}
% 量和单位
\usepackage{siunitx}

% 参考文献使用 BibTeX + natbib 宏包
% 本科生参考文献的著录格式
\usepackage[sort]{natbib}
\bibliographystyle{thuthesis-bachelor}

% 定义所有的图片文件在 figures 子目录下
\graphicspath{{figures/}}

% 数学命令
\makeatletter
\newcommand\dif{%  % 微分符号
  \mathop{}\!%
  \ifthu@math@style@TeX
    d%
  \else
    \mathrm{d}%
  \fi
}
\makeatother

% hyperref 宏包在最后调用
\usepackage{hyperref}
% 多个引用
\usepackage{cleveref}
\crefname{theorem}{定理}{定理}
\crefname{lemma}{引理}{引理}
\crefname{definition}{定义}{定义}
\crefname{figure}{图}{图}
\crefname{table}{表}{表}
\crefname{algorithm}{算法}{算法}
\crefname{equation}{公式}{公式}
\crefname{section}{}{}
\newcommand{\crefrangeconjunction}{至}